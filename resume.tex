\documentclass[10pt,letterpaper,sans]{moderncv}

\usepackage[scale=0.75]{geometry}

\moderncvstyle{casual}
\moderncvcolor{blue}

\name{Alec}{Heller}
\address{2 Berwick St. \#2}{Somerville, MA}
\phone[fixed]{530-628-3773}
\email{alec@deviant-logic.net}
\social[github]{deviant-logic}
\social[linkedin]{alecheller}
% \quote{Hire me, I know stuff}

\begin{document}
\makecvtitle

\section{Experience}

\cventry{2017--2022}
        {Staff Engineer, Architect}
        {SimSpace Corporation}
        {Boston, MA}
        {}
        {%
          \begin{itemize}
            \item Designed and implemented hypervisor abstraction and network building platform
              capability.
            \item Designed and implemented \texttt{avaleryar}, an open-source domain specific
              language for expressing sophisticated access-control and authorization policies.
              Integrated same across numerous platform components.
            \item Designed and implemented "Live Action Events", a large-scale security training
              product coordinating tens to hundreds of participants in simulated network attacks.
            \item Designed and implemented "Content Repository", a platform service providing
              heterogeneous content search for both SimSpace and non-SimSpace authored content.
              Along with "Multi-Tiered Organizations" (effectively a directory service), Content
              Repo implemented hierarchical content sharing.
            \item Designed and implemented content distribution protocols and capabilities.
            \item Architect: Provided consultation, mentorship, design and implementation to
              stakeholders across engineering, including feature development, maintenance,
              devops and SRE.
          \end{itemize}
        }

\cventry{2016--2017}
        {Senior Developer}
        {Lexumo}
        {Burlington, MA}
        {}
        {%
          Feature development, system architecture, database design.  ETL pipeline design
          and implementation.
        }

\cventry{2014--2016}
        {Haskell Consulting}
        {XKCD and Rein Heinrichs Consulting}
        {Cambridge, MA}
        {}
        {%
          \begin{itemize}
          \item Designed and implemented the Chronicle Time-Series Database.
          \item Designed and implemented the \texttt{arena} durable storage and indexing system.
          \item Designed and implemented Haskell bindings for the Fastly CDN's configuration API.
          \item Designed and implemented the pure Haskell \texttt{miss} library for interacting with
            Git repositories.
          \end{itemize}
        }

\cventry{2010--2014}
        {Senior Software Engineer}
        {Akamai Technologies}
        {Cambridge, MA}
        {}
        {%
          \begin{itemize}
          \item Developed, ported, and maintained the Content Adaptation Engine (Perl, Clojure).
          \item Designed and implemented Edge Device Characterization (C++, Lex, Protocol Buffers).
          \item Supported EDC via large-scale data analysis (Hadoop, Ruby, Haskell, Prolog).
          \item Designed and initiated implementation on the Resource Timing Analytics Framework
            (Spark, Mesos, Docker).
          \item Designed the Real User Monitoring Reporting API (REST, JSON, BigTable).
          \item Designed and implemented the test environment for the Certificate Provisioning
            System (Vagrant, Puppet, Postgres, Apache, OpenSSL).
          \item Researched and proposed (work ongoing at the end of my tenure) authentication
            framework for the Certificate Provisioning System (TLS, SSH, Cat Herding).
          \end{itemize}
        }

\cventry{2005--2009}
        {Instructor, Research Assistant}
        {Northeastern University}
        {Boston, MA}
        {}
        {%
          \begin{itemize}
          \item Teaching: Undergraduate Java and Scheme, Graduate Scheme, and non-major intro.
          \item Research in Substructural Type Systems, Session Types, Program Logics, and Proof
            Assistants.
          \end{itemize}
        }

\cventry{Summer, 2007}
        {Ocaml Summer Project}
        {Jane Street Capital}
        {New York, NY}
        {}
        {%
          \begin{itemize}
          \item Designed and implemented \texttt{caml-shcaml}, an Ocaml library for Unix shell
            programming (Ocaml, Unix).
          \end{itemize}
        }

\cventry{Summer, 2006}
        {Senior Intern}
        {Grammatech}
        {Ithaca, NY}
        {}
        {%
          \begin{itemize}
          \item Contributed to the prototype implementation of a system for inferring and
            model-checking software library protocols (i.e., no \texttt{write} after
            \texttt{close}).
          \end{itemize}
        }

\cventry{2003--2004}
        {Programmer, Sysadmin}
        {Transaction Auditing Group}
        {New York, NY}
        {}{}

% \subsection{Consulting}

% \cventry{2004--2010}
%         {Haskell Development}
%         {iPwn Studios}
%         {Roxbury, MA}
%         {}{}
% \cventry{}
%         {Ruby on Rails Development}
%         {Babel Research}
%         {Lincoln, MA}
%         {}{}
% \cventry{}
%         {Ruby on Rails Development}
%         {Milestone Capital Management}
%         {Greenwich, CT}
%         {}{}

\section{Skills}

\cvitem{Technical}{Language Technology, Unix, Software Design, Good Taste,
  Algorithmics.}

\cvitem{Formal Languages}{Awk, C++, C, Clojure, Coq, Erlang, Haskell, Java,
  Ocaml, Perl, Prolog, R, Racket, Ruby, SML, Scala, Scheme, Sed, Sh,
  Twelf.}

\cvitem{Natural Languages}{Native English, Passable Spanish, Cursory ASL.}

\cvitem{Musical Instruments}{Piano, Guitar, Bass Guitar.}

\pagebreak
\section{Details On Recent Projects}

\cvitem{Avaleryar Authorization Policy Language (SimSpace, Open Source)}{An implementation of
  Pimlott and Kiselyov's \emph{Soutei} system for expressing and validating rules-based
  access-control policies.  The code is available at \url{https://github.com/Simspace/avaleryar}.}

\cvitem{Arena Durable Storage and Indexing Library (October 2015--May 2016)}{While the work on
  Chronicle remains proprietary, the library which undergirds its durable storage and indexing
  system is open source and available at \url{http://hackage.haskell.org/package/arena}.}

\cvitem{Fastly API Bindings (October 2015)}{Implemented bindings to the Fastly HTTP API.  This was a
  quick implementation of a large (and somewhat persnickety) API.  The code is available at
  \url{http://code.xkrd.net/alec/fastly}, and induced a useful utility library for JSON processing
  (\url{http://hackage.haskell.org/package/aeson-filthy}).}

\cvitem{Chronicle Time Series Database (July 2015--May 2016)}{Designed and implemented a lightweight
  event processing/time series database for logging, querying and archiving dashboard alerts.}

\cvitem{Pure Haskell Git Library (May 2015--Present)}{Because we wanted Skete to be reasonably
  portable, we wanted to avoid depending on bindings to @libgit2@ or similar.  This afforded me the
  chance to implement a pure-Haskell library for interacting with Git repositories
  (\url{http://code.xkrd.net/skete/miss}).}

\cvitem{Skete: An Approach to Distributed Package Management (December 2014--April 2015,
  ongoing)}{Skete is ongoing work to develop a distributed package management system---effectively a
  ``Hackage meets Git''.  We provide both a supporting library
  (\url{http://code.xkrd.net/skete/skete}) and a Hackage-specific instantiation
  (\url{http://code.xkrd.net/skete/skete-haskell}).  Currently, we actually use Git as our storage
  backend.}

\cvitem{Resource Timing Analytics Framework}{Designed and initiated
  implementation (it's work in progress) of a system for ad hoc
  analysis of massive quantities of web page resource timing
  measurements.  The system provides a flexible interface for querying
  the data via Dataflow, SQL, and MapReduce-style programs, in
  arbitrary programming languages/environments.  Architecture
  components: Apache Spark, Mesos, Docker.}

\cvitem{Real User Monitoring Reporting API}{Designed, specified, and
  coordinated the API bridging front-end report rendering and back-end
  report generation (and the 3 different software teams, one remote,
  responsible for same) for Akamai's RUM Reporting feature.  The API
  provides an organized, extensible framework for a whole class of
  reports analyzing detailed breakdowns of web page performance.  It's
  withstood 2 (soon to be 3) back-end architectures so far.  Relevant
  keywords: Big Data, REST, query languages, JSON, BigTable.}

\cvitem{Edge Device Characterization}{Designed and implemented
  Akamai's high-performance device characterization capability, which
  is the technology behind two products and several different in-house
  data analysis systems.  EDC is capable of identifying over 100,000
  user agents per second while running in 15MB of core.  It has also
  been running bug free for years.  Main implementation: C++,
  Lex, Protocol Buffers.  Supporting software: C, Haskell, Ruby,
  Prolog, Make.}

\section{Education}

\cventry{2005--2009}
        {Master of Science}
        {Northeastern University}
        {Boston, MA}
        {Programming Languages}
        {}
\cventry{1999--2003}
        {Bachelor of Arts}
        {Oberlin College}
        {Oberlin, OH}
        {Computer Science}
        {}

\section{Publications and Talks}

\cvitem{}
 {\emph{Systems and methods for identifying and characterizing client devices},
   US Patent 13730528, assigned August 16, 2016.}
\cvitem{}
 {\emph{UA strings are terrible: adventures in server-side device characterization},
   O'Reilly Velocity Conference, Santa Clara, 2014.}
\cvitem{}
 {\emph{Prolegomenon to any future talk on hacking ghc},
  Presented to the Boston Haskell Users Group, 2009.}
\cvitem{}
 {\emph{Caml-shcaml: an ocaml library for unix shell programming} (with Jesse Tov),
  Proceedings of the 2008 ACM SIGPLAN workshop on ML.}
\cvitem{}
 {\emph{Session types made boring},
  Presented at the New England Programming Languages Seminar, 2007.}

\end{document}
