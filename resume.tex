\documentclass[10pt,letterpaper,sans]{moderncv}

\usepackage[scale=0.75]{geometry}

\moderncvstyle{casual}
\moderncvcolor{blue}

\name{Alec}{Heller}
\address{2 Berwick St. \#2}{Somerville, MA}
\phone[fixed]{530-628-3773}
\email{alec@deviant-logic.net}
\social[github]{deviant-logic}
\social[linkedin]{alecheller}
% \quote{Hire me, I know stuff}

\begin{document}
\makecvtitle

\section{Experience}

\subsection{Employment}

\cventry{2010--2014}
        {Senior Software Engineer}
        {Akamai Technologies}
        {Cambridge, MA}
        {}{}

\cventry{2005--2009}
        {Instructor, Research Assistant}
        {Northeastern University}
        {Boston, MA}
        {}{}

\cventry{Summer, 2007}
        {Ocaml Summer Project}
        {Jane Street Capital}
        {New York, NY}
        {}{}

\cventry{Summer, 2006}
        {Senior Intern}
        {Grammatech}
        {Ithaca, NY}
        {}{}

\cventry{2003--2004}
        {Programmer, Sysadmin}
        {Transaction Auditing Group}
        {New York, NY}
        {}{}

\subsection{Consulting}

\cventry{2014--2016}
        {Haskell Development}
        {XKCD and Rein Heinrichs Consulting}
        {Cambridge, MA}
        {}{}
\cventry{2004--2010}
        {Haskell Development}
        {iPwn Studios}
        {Roxbury, MA}
        {}{}
\cventry{}
        {Ruby on Rails Development}
        {Babel Research}
        {Lincoln, MA}
        {}{}
\cventry{}
        {Ruby on Rails Development}
        {Milestone Capital Management}
        {Greenwich, CT}
        {}{}

\section{Technical Skills}

\cvitem{}{Language Technology, Unix, Software Design, Good Taste,
  Algorithmics}

\cvitem{Languages}{Awk, C++, C, Clojure, Coq, Erlang, Haskell, Java,
  Ocaml, Perl, Prolog, R, Racket, Ruby, SML, Scala, Scheme, Sed, Sh,
  Twelf.}

\section{Neat Projects I've Done Recently}

\cvitem{Edge Device Characterization}{Designed and implemented
  Akamai's high-performance device characterization capability, which
  is the technology behind two products and several different in-house
  data analysis systems.  EDC is capable of identifying over 100,000
  user agents per second while running in 15MB of core.  It has also
  been running bug free for over 2 years.  Main implementation: C++,
  Lex, Protocol Buffers.  Supporting software: C, Haskell, Ruby,
  Prolog, Make.}

\cvitem{Resource Timing Analytics Framework}{Designed and initiated
  implementation (it's work in progress) of a system for ad hoc
  analysis of massive quantities of web page resource timing
  measurements.  The system provides a flexible interface for querying
  the data via Dataflow, SQL, and MapReduce-style programs, in
  arbitrary programming languages/environments.  Architecture
  components: Apache Spark, Mesos, Docker.}

\cvitem{Real User Monitoring Reporting API}{Designed, specified, and
  coordinated the API bridging front-end report rendering and back-end
  report generation (and the 3 different software teams, one remote,
  responsible for same) for Akamai's RUM Reporting feature.  The API
  provides an organized, extensible framework for a whole class of
  reports analyzing detailed breakdowns of web page performance.  It's
  withstood 2 (soon to be 3) back-end architectures so far.  Relevant
  keywords: Big Data, REST, query languages, JSON, BigTable.}

\cvitem{Chronicle Time Series Database}{Designed and implemented a lightweight
  event processing/time series database for logging, querying and archiving
  dashboard alerts.  While this work remains proprietary, the library which
  undergirds its durable storage and indexing system is open source and
  available at \url{http://hackage.haskell.org/package/arena}.}

\cvitem{Skete: An Approach to Distributed Package Management}{Skete is ongoing
  work to develop a distributed package management system---effectively a
  ``Hackage meets Git''.  We provide both a supporting library
  (\url{http://code.xkrd.net/skete/skete}) and a Hackage-specific instantiation
  (\url{http://code.xkrd.net/skete/skete-haskell}).  Currently, we actually use
  Git as our storage backend.  This provided an opportunity to implement a
  pure-Haskell library for interacting with Git repositories
  (\url{http://code.xkrd.net/skete/miss}).}

\cvitem{Fastly API Bindings}{Implemented bindings to the Fastly HTTP API.  This
  was a quick implementation of a large (and somewhat persnickety) API.  The
  code is available at \url{http://code.xkrd.net/alec/fastly}, and induced a
  useful utility library for JSON processing
  (\url{http://hackage.haskell.org/package/aeson-filthy}).}

\section{Education}

\cventry{2005--2009}
        {Master of Science}
        {Northeastern University}
        {Boston, MA}
        {}
        {Programming Languages}
\cventry{1999--2003}
        {Bachelor of Arts}
        {Oberlin College}
        {Oberlin, OH}
        {}
        {Computer Science}

\section{Publications and Talks}

\cvitem{}
 {\emph{UA strings are terrible: adventures in server-side device characterization},
   O'Reilly Velocity Conference, Santa Clara, 2014.}
\cvitem{}
 {\emph{Prolegomenon to any future talk on hacking ghc},
  Presented to the Boston Haskell Users Group, 2009.}
\cvitem{}
 {\emph{Caml-shcaml: an ocaml library for unix shell programming} (with Jesse Tov),
  Proceedings of the 2008 ACM SIGPLAN workshop on ML.}
\cvitem{}
 {\emph{Session types made boring},
  Presented at the New England Programming Languages Seminar, 2007.}

\section{Natural Languages}

\cvitem{English}{Native}
\cvitem{Spanish}{Passable}
\cvitem{American Sign Language}{Cursory}

\section{Personal}

\cvitem{Music}{Jazz Piano, Guitar, Bass.}
\cvitem{Cooking}{Mostly improvisational, usually tasty.}

\end{document}
